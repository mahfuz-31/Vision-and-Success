\chapter{The \textit{Sahaaba} (companions) Knew}
\textvtt{
We previously discussed that, success is to live a life that will lead us to \textit{Jannah}. \textbf{No one understood this better than the companions of the Prophet peace and blessings be upon him}.

When the Prophet peace and blessings of Allaah be upon his soul came to \textit{Madina}, the inhabitans of the blessed city of \textit{Madina} (the \textit{ansaar}) opened their hearts, doors, houses and lives for the immigrants that came from \textit{Makkah}. They put an example of sacrifice that is still not found in the history of our 1400 years. Let me give an example of what hospitality that they extended to these immigrants. 

A companion named Abdurrahman Ibn 'Awuf may Allaah be pleased with him came from \textit{Makkah}, and the disbelievers of \textit{Makkah} took everything from him. They took his wife, children, wealth, mount and he walked through the 450 kilometers of desert in an accute heat alone. When he arrived in \textit{Madina}, the \textit{Rasul} peace be upon his soul had just come out of the masjid and saw the man Abdurrahman Ibn 'Awuf whom he knew from \textit{Makkah} from his own tribe \textit{Quraish}. Normally when anyone travels, he brings luggage, mount and other stuffs, but the Prophet peace be upon his soul saw that Abdurrahman is alone and his lipes were cracked and feets were blistered. So the \textit{Rasul} said, "What did you bring?". He said, "Nothing! O Prophet of Allaah. I lost everything." The Prophet peace be upon his soul said, "No, you gained \textit{Jannah}." Then the Prophet of Allaah peace be upon him called an \textit{ansari} named Sa'ad Ibn 'Ubaadah may Allaah be pleased with him and said, "From today you are brother with this \textit{muhajir} (immigrant)". Sa'ad Ibn 'Ubaadah may Allaah be pleased with him took him and said, "Come my brother, I've two houses. Look, which one you like, I will vacate it for you. I've two businesses. Look at which one you like, I will hand the keys of that over to you. (and to the extend, \textit{Subhanallah}!) Come, I've two wives, look at which one you like, I'll divorce her. Wait for her \textit{'iddah} to finish. Then marry her".

They showed an example of hospitality not to be surpassed in history. These were the \textit{ansaar}. \textit{Islam} was indebted to them, the \textit{muhajirin} were indebted to them. And not only this, when it came time for campaigns and battles that had to be fought in the defense of the \textit{deen}, it was the \textit{ansaar}, it was their men and their arms. And the Prophet peace be upon him watched those \textit{ansaar} doing all these. There were battles in the Islamic history in which there was not a single \textit{muhajir} in it, the \textit{ansari} went by himself, finished the job and came back.

The Prophet peace and blessings of Allaah be upon his soul was longing for an opportunity that, somehow he could repay their favour. But the \textit{ansaar} never asked. They never said that, O Prophet, give us this and that. Eventually one day these blessed individuals (the \textit{ansaar}) faced a problem. There was a well which was at the outside of the city of \textit{Madina} for the irrigation and feeding of the livestocks. And they had to go out and get the water for their use, and it was tedious for them. A young man had a dazzling idea. He tells the others that, "Listen, there's an easy solution to this. We go the Rasul peace and blessings be upon his soul and say, O Messenger of Allaah, make dua that Allaah \textit{rabbul 'izzah} open for us water in the middle of the city like He did for \textit{Zamzam} in the middle of \textit{Makkah}". So they said, brilliant idea! All the problems will be solved and there will be ease. Now they entered the masjid in their group and on their face you could see that a twinkle of request is going to come. So the Rasul peace be upon him knew from their face and said, "Glad tidings to \textit{ansaar}! Today whatever they ask me I will give". They heard the Prophet peace be upon him saying this as they were walking to him. And all of a sudden the well became too small. Like everything else wasn't guaranteed, khalas we would ask for the well, but now anything in the heavens and earth is guaranteed. By the Lord of the \textit{Ka'aba}! If the Rasul were to ask for \textit{Uhud} to turn to gold, Allaah would have made \textit{Uhud} into gold. So they realized that it's a blank check, anything we ask it will be given. So the young man said, "O Prophet of Allaah! give us a moment, so that we can discuss". They discussed with one another. They all said to themselves, "Listen, forget about the well. Let's ask the Prophet to ask Allah to firgive us". So they came to Prophet of Allaah and said, "O Messenger of Allaah! Ask Allaah to forgive us". So the Rasul peace and blessings be upon him raised his hand and asked,
\begin{center}
    \textit{
        O Allaah! Forgive the ansaar.\\
        And the children of the ansaar.\\
        And the children of the children of ansaar
    }
\end{center}
They shouted out, "And our servents O Prophet of Allah! And our servents O Prophet of Allah!" So the Rasul of Allah added, 
\begin{center}
    \textit{And their servents.}
\end{center}

Do you see! that they understood, the greatest success is the success of the hereafter. As Allah stated in His book,
}
\begin{center}
    \begin{RLtext}
        قَالَ ٱللَّهُ هَٰذَا يَوْمُ يَنفَعُ ٱلصَّٰدِقِينَ صِدْقُهُمْۚ لَهُمْ جَنَّٰتٌ تَجْرِى مِن تَحْتِهَا ٱلْأَنْهَٰرُ خَٰلِدِينَ فِيهَآ أَبَدًاۚ رَّضِىَ ٱللَّهُ عَنْهُمْ وَرَضُوا۟ عَنْهُۚ ذَٰلِكَ ٱلْفَوْزُ ٱلْعَظِيمُ 
    \end{RLtext}
\end{center}
\textbf{
    Allāh will say, "This is the Day when the truthful will benefit from their truthfulness." For them are gardens [in Paradise] beneath which rivers flow, wherein they will abide forever, Allāh being pleased with them, and they with Him. That is the great success.} [5:119]

\textvtt{
They could have asked for anything. Look at another companion, so that you don't think it's an abstract incident.

There was a companion, he used to watch that, people were always arround the Rasul peace be upon him during the day at His service. At night, it wasn't like our time, it used be dark all around and no one used to come out of home most often. When at night everyone went home, this companion of the Prophet peace be upon him thought that, now who would serve the Messenger of Allaah if He asks for something, or if he need anything. Without making a fuss, he came and sat by the door of the Prophet, all night. And he did this one day, two days. Then eventually the Prophet peace be upon him came out of home at one night and He saw him sitting. So He asked him, "What are you doing here at this hour?". The man explained, "O Messenger of Allaah! during the day you have everyone at your service. At night there's no one, I thought if the Prophet needs something, there should be someone. So I came and sit here." He was unspoken quitely sitting, he didn't came and knocked on the door, "O Prophet of Allah, I'm sitting here, ask me if you need anything." No, he didn't do that. So it touched the heart of the Prophet peace be upon his soul. He said, 
\begin{center}
    \begin{RLtext}
        سل, 
    \end{RLtext}
    \textit{Ask, It will be granted.}
\end{center}
The man asked some time to the Prophet of Allaah to think about this. Then after a while he asked, "O Messenger of Allaah! I want your companionship in \textit{Jannah}." The Prophet peace be upon him asked, "Who told (taught) you this?" He said, "Allaah put it in my heart". Then the \textit{Rasul} said, "Increase in your prostration so that it makes the process of my du'a easy". As if the Prophet peace be upon him is asking him to increase his obedience of Allaah, as Allaah said, 
}
\begin{center}
    \begin{RLtext}
        وَمَن يُطِعِ ٱللَّهَ وَرَسُولَهُۥ فَقَدْ فَازَ فَوْزًا عَظِيمًا 
    \end{RLtext}
    \textbf{And whoever obeys Allāh and His Messenger has certainly attained a great attainment.} [33:71]
\end{center}
